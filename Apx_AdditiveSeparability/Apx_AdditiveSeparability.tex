\documentclass{article}
\usepackage{graphicx} % Required for inserting images
\usepackage{amsmath}
\usepackage{bbm} %para 1 indicatriz
\usepackage{cancel}
\usepackage{enumitem} %fancy itemize

\title{formal}
\author{leontaglelb }
\date{July 2024}

\begin{document}



The main difference between our equilibrium and that of \cite{schnell2017physician} is the presence of a ``strategic effect'', wherein doctor $j$ takes into account the behavior of other doctors in the selection of her own $\bar{\kappa_j}$. Of the different modifications we made to her framework, we'll argue it's the absence of \textit{additive separability} across patients in physician utility $U_j(\cdot)$ which accounts for this.

In our context of unbounded maximization by doctor $j$, the absence of additive separability implies she can't consider each patient individually when it comes to whether she's willing to allow (or induce) their visit, as such a decision is no longer independent of other patients' visits; the marginal utility of an additional patient is dependent on the aggregate of clients up to that point, both in terms of the visit itself as well as in the number of sick leave certificates granted up to that point.




Let’s illustrate this point. Consider for a moment a finite number of patients $1, ... , k$, where each patient is inputed directly as an argument in our doctor $j$’s $U_j(\cdot)$, like so: $U_j(1, ..., k)$. If $U_j(\cdot)$ has the property of additive separability, this means it may be reformulated like so:
\[
U_j(1, ...,k) = v_{j1}(1) + ... + v_{jk}(k) = \sum_{i = 1}^{k} v_{ji}(i)
\]
Unconstrained optimization in this context implies she’s willing to see any patient whose $v_{ji}(i)$ is non-negative, such that her optimal level of utility is:
\[
U_j^*(1, ...,k) =  \sum_{i : \; v_{ji}(i) \geq 0} v_{ji}(i)
\]
In the context of our doctor-patient model, where physician utility is increasing in $\kappa_i$, selection is achieved by the doctor by choosing a $\kappa_j^*$ which excludes all patients $i$ whose $\kappa_i < \kappa_j^*$. Supposing our patients are well-ordered in $\kappa_i$, the choice of such a $\kappa_j^*$ would be one where the marginal consumer $i$ affords a non-negative $v_{ji}(i)$, and the inframarginal consumer $i + 1$ fulfills $v_{j,i+1}(i+1) < 0$. Ignoring for a moment that patients themselves have a \textit{choice} of visiting -- depending upon a second dimension $\gamma$ --, we would then have:
\[
U_j^*(1, ...,k) =  \sum_{i : \; \kappa_i \geq \kappa_j^*} v_{ji}(i)
\]

If instead of a discrete set we consider a mass of consumers $\mathcal{I}$ characterized by their level of $\kappa_i$, and we make the simplifying assumption that $v_{ji}$ takes on the same form $v_{j}$ for every $i \in \mathcal{I}$, our $U_j(\cdot)$ could be expressed as:
\[
U_j^*(\mathcal{I}) =\int_{\kappa_j^*}^{\infty} v_{j}(k) \, dF(k)
\]

$U_j^*(\mathcal{I})$ represents the optimal value of $U_j(\mathcal{I})$, where doctor $j$ only sees patients who provide her with non-negative marginal utility, i.e. such that $\kappa_i \geq \kappa_j^*$. We can find this optimal value $\kappa_j^*$ by looking at \textit{threshold} equilibria, where doctor utility is also dependent on the threshold $\bar{\kappa_j}$ over which she’s willing to see patients:
\[
U_j(\mathcal{I}, \bar{\kappa_j}) =\int_{\bar{\kappa_j}}^{\infty} v_{j}(k) \, dF(k)
\]

The optimal value of threshold $\bar{\kappa_j}$ is $\kappa_j^*$. Assuming $U_j(\cdot) $ is twice differentiable and concave in $\bar{\kappa_j}$, such a solution may be arrived at through the FCO:
\[
\frac{\partial U_j(\mathcal{I})}{\partial \bar{\kappa_j}} \equiv - v_{j}(\bar{\kappa_j})f(\bar{\kappa_j}) = 0
\]
Solving for this would yield $\bar{\kappa_j} = \kappa_j^*$.

\cite{schnell2017physician} is an example of just such a treatment, which specifies the physician's additive utility by patient in the following manner:
\[
v_{ji}(\kappa_i) \equiv R_j + \beta_j h(\kappa_i)
\]
where $R_j$ is a parameter standing for revenue by visit, and $\beta_j h(\kappa_i)$ represents the physician's ``altruistic'' utility over the health impact of a prescription drug to a patient with ``pain level'' $\kappa_i$.

The optimal threshold $\kappa_j^*$ is then obtained out of the maximization\footnote{Once again, ignoring patient choice and $\gamma$,. We're presenting a \textit{simplified} version for illustrative purposes.}:
\[
\max_{\bar{\kappa_j}}\int_{\bar{\kappa_j}}^{\infty}  R_j + \beta_j h(k) \, dF(k)
\]

Which gives out the following FOC:
\[
R_j = - \beta_j h^{\prime}(\bar{\kappa_j})
\]
which, as is immediately apparent, doesn't depend upon the behavior of other doctors ---more specifically, \textit{their} choice of $\bar{\kappa_j}$. It merely establishes that, at the threshold, marginal benefit by patient --revenue $R_j$-- must equal marginal ``cost'' -- altruistic ``cost'' $\beta_j h(\bar{\kappa_j})$--. A way to interpret this is that doctors will be willing to see any and all patients which render them positive marginal utility, unconcerned with their total market share, and thus, what other physicians may do to take away clientele.

Such a treatment is rendered inviable by our choice of utility function. Schnell's parameter of revenue would in our model imply the linearity of our revenue \textit{function} $R_j(\cdot)$ and our $P(\cdot)$ function over \textit{aggregate} licenses granted. Strict convexity of $P(\cdot)$ would forestall its formulation as Schnell-like $\beta_j h^{\prime}(\bar{\kappa_j})$ terms for each patient, because the impact on doctor $j$'s utility in granting patient $i$ a license would no longer independent from the granting of licenses of other patients. Likewise, strict concavity of $R_j(\cdot)$ belies a simple ``r'' parameter such that $R_j'(Q_j) = r Q_j$.

More formally, when $U_j(1,...,k)$ \textit{isn't} additively separable across patients, the value $\kappa_j^*$ such that if $\kappa_i \geq \kappa_j^*$ patient $i$ provides positive marginal utility, isn’t independent of current clientele, because what before was a properly defined object, marginal utility by patient $i$, $v_j(\kappa_i)$, can no longer be so identified. The marginal utility $i$ provides to $j$ as the $k$th client (assuming some order over clients) is not necessarily the same he’d provide as the $k+1$th client, and so, as the $k$th client he could provide $0$ utility, impliying $\kappa_i = \kappa_j^*$, whereas as the $k + 1$th he could be inframarginal, such that $\kappa_i < \kappa_j^*$.

$\kappa_j^*$ is not longer independent of clientele mass $\mathcal{I}$ as before, but a function of it, $\kappa_j^*(\mathcal{I})$. More specifically, in the models we consider it will depend on the \textit{cardinality} of current clientele, $|\mathcal{I}|$, such that our physician's choice of marginal consumer will depend on her aggregate level of patient demand, where before it didn't. This effect is introduced through the \textit{strict non-linearity} of our physician utility function $U_j(\cdot)$, either through the strict concavity of $R_j(\cdot)$ over expected patient demand $Q_j$, or the strict convexity of $P_j(\cdot)$ over total expected sick leaves granted.

When either of those is the case, $aggregate$ levels enter into the equation and form part of the optimality condition. Our general FOC reflects this:
\begin{equation*}
 R_j^{\prime}(Q_j)\frac{\partial Q_j}{\partial\bar{\kappa_j}} = P_j^{\prime}(X_j)\frac{\partial X_j}{\partial \bar{\kappa_j}} 
\end{equation*}
where either or both of $R_j'(\cdot)$ and $P_j'(\cdot)$ are a non-constant function over aggregates.

When such aggregates, either $Q_j$ or $X_j$, come into play, physicians come to consider their \textit{market share}, which doesn't depend exclusively on their own choice of $\bar{\kappa_j}$. Each patient's strategy $S_i$ is constructed taking into account the whole of $\{(V_j,\bar{\kappa_j})\}_{i =1}^{J})$.


ALGO ACERCA DE COMO SE SALE DE EQUILIBRIO, LELVA A PENSAR QUE Y CONCLUYE

\subsection{som}

To prove our point that it is additive separability which accounts for a possible strategic effect, we reformulate \cite{schnell2017physician} in the manner of a McFadden Logit -- with some quirks.

The way in which Schnell devises her physician's utility formula is implicitly Bernoulli-like:
\[
\int_{\kappa} \int_{\gamma} u(\kappa, \gamma) \cdot p(\kappa, \gamma) \, dG(\gamma) dF(\kappa)
\]
where $u(\kappa, \gamma)$ is the utility function of the physician from patients characterised by a given $(\kappa, \gamma)$ tuple, and $p(\kappa, \gamma)$ stands for the proportion of clients atomized in such a tuple the physician expects to have visit her.

$u(\kappa, \gamma)$ we'll leave as Schnell defined it: $\beta_j h(\kappa) + R_j$. $p(\kappa, \gamma)$ lends itself nicely as the $s_{ij}$ we have been using thus far, which implicitly depends on the $(\kappa_i, \gamma_i)$ tuple which characterizes patient $i$:
\[
\int_{\kappa} \int_{\gamma} [\beta_j h(\kappa) + R_j] \cdot s_{ij}(\kappa, \gamma) \, dG(\gamma) dF(\kappa)
\]


In our modeling section, our ``implicit'' search model was a left-censored Logit choice model, which gave null probability of assignment to physicians from which patient $i$ expected non-positive utility. It was defined as:
\[
s_{ij} = \frac{\alpha_{ij}}{\sum_{k = 1}^{J} \alpha_{ik}}, \; \; \text{where } \; \alpha_{ij} = \begin{cases}
e^{\lambda u_{ij}}, \; \; \text{if } \; \; u_{ij} > 0 \\
0 , \; \; \text{if } \; \; u_{ij} = 0
\end{cases}
\]

Schnell's opioid-focused physician model provides an additional simplification: patients which aren't given a drug prescription, i.e. those such that $\kappa_i < \bar{\kappa_j}$, don't garner positive utility from a visit. As such, in the original model as in this reformulation, patients ``below'' $\bar{\kappa_j}$ have a null probability of visit, such that the inferior limit of integration for $\kappa$ is $\bar{\kappa_j}$.

As for $\gamma$, the limit of the inner integral remains the same as in Schnell in the absence of a secondary market, a value of $\gamma_i$ such that $u_{ij} = 0$, i.e. $h(\kappa_i) + \gamma_i - \tau^d - \tau^o = 0$.\footnote{Schnell splits the costs a patient will face into costs of visit ($\tau^d$), cost of purchase ($\tau^o$) and search cost ($\tau^s$)}

The double-integral maximized as physician utility is then as follows:
\[
\max_{\bar{\kappa_j}}\int_{\bar{\kappa_j}}^{\infty} \int_{\tau^d - \tau^o - h(\kappa)}^{\infty} [\beta_j h(\kappa) + R_j] \cdot \frac{e^{\lambda u_{ij}}}{\sum \limits_{k : \, u_{ik} > 0} e^{\lambda u_{ik}}} \, dG(\gamma) dF(\kappa)
\]

The FOC of such an equation is:
\begin{equation}
[\beta_j h(\bar{\kappa_j}) + R_j]\int_{\tau^d - \tau^o - h(\bar{\kappa_j})}^{\infty} \frac{e^{\lambda u_{ij}} \mid \bar{\kappa_j}}{\sum \limits_{k : \, u_{ik} > 0} e^{\lambda u_{ik}} \mid \bar{\kappa_j}} \, dG(\gamma) f(\bar{\kappa_j}) = 0 
\label{eq:schnell_logit}
\end{equation}


Where we use ``$\mid \bar{\kappa_j}$'' to clumsily indicate that we integrate $\gamma$ over patients where $\kappa_i = \bar{\kappa_j}$. As an integral over a strictly positive value and the atom of a density function, respectively, the factors B and C in the equation are non-negative. For this reason, the only way for (\ref{eq:schnell_logit}) to be fulfilled is the following condition:
\begin{equation}
    R_j = - \beta_j h(\bar{\kappa_j})
    \label{eq:schnell_condition}
\end{equation}

This is the same result as in the original model in \ref{schnell2017physician} in the absence of a secondary market, a condition which stipulates that marginal utility in $\bar{\kappa_j}$ must be $0$, i.e. that revenue ($R_j$) must equal "altruistic" loss ($\beta_j h(\bar{\kappa_j})$), which pre-supposes that the marginal patient granted prescription opioids suffers a net loss in utility (the negative externalities outweigh its medical benefit as a palliative).

Condition (\ref{eq:schnell_condition}) means a value  $\kappa_j$ is chosen irrespective of the strategies employed by other doctors, it quite literally doesn't enter into the equation. As we have argued, additive separability over clients means the \textit{total} demand for physician $j$'s services doesn't influence marginal utility by a given patient $i$, and so has no sway in the optimality condition. Doctor $j$ states simply that she won't grant a prescription (or sick leave, as in our case) below $\bar{\kappa_j}$, \textit{come what may}. In order for her to care about \textit{aggregate} values, like her total expected demand, making her care about her market share and thus about the strategies of other doctors, \textit{strict non-linearity} must be introduced into her utility function.

\end{document}
