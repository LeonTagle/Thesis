\documentclass[../main.tex]{subfiles}

\begin{document}

\section{Literature Review}


It is a well established factum that taking sick leave is subject to an economic calculation on part of the workers, rather than being an orthogonal, merely health-concerned matter. \cite{JohanssonPalme} begin their article with a quote by Nobel Laureate Ragnar Frisch: ``Regarding the high absence rate at the Department: Acquiring minor diseases, such as colds or flu, is an act of choice''. Their paper is among many others --\cite{italian-jobs}, \cite{norway-jobs}, \cite{us-jobs}, \cite{sweden-jobs}-- which give empirical evidence of such a choice being driven by economic incentives, through an event study on exogenous institutional regime changes in the subject nation's public insurance system. This line of research, though, is concerned with the actions of workers themselves and their subsequent effect on macroeconomic employment variables, whereas our main focus shall be the role played by physicians.

Our doctors' utility function is composed of two terms: one concerning revenue, the other patients' health. This is in line with the literature on physicians, which now commonly regards them as ``altruistic'' agents whose utility is to a higher or lesser degree dependent on that of the patients, a claim which has found empirical support in both medical students (\cite{avengers}, \cite{hs-wiesen}) as well as doctors themselves (\cite{hippocrates}, \cite{brosigkoch}). \cite{crea2019physician} finds no evidence for this, whereas \cite{godager2013profit} do, and explore its heterogeneity across physicians. The fact that physicians are also concerned with revenue, rather than being purely altruistic, is also well evidenced, see \cite{clemensgottlieb}, \cite{HSW}, \cite{autor}, and also \cite{rrk2012} for a review on the matter. Therein lies the dilemma with giving physicians the status of gatekeepers for different services and certifications, like disability insurance (as in autor). As \citet[p.~1]{markussen-roed} put it: ``In essence, the GPs [general practitioners] have been assigned the task of protecting the public (or private) insurer's purse against the customers who form the basis for their own livelihood''.

Both factors being well established in the literature on physicians, one strand of it would seek to design an optimal contract for medical care in the presence of such an economic calculus, see \cite{chone-ma} or \cite{optimal-altruism}; another, more in line with our approach, would evaluate the effects increased competition among doctors has on their rendered services. In general, \cite{currie2023effects} propose that increased competition would lead to physicians offering more services that please the clients yet relatively hurt their own utility (like drug prescriptions), and less services which bring them, physicians, more utility, at the expense of patient utility (like unwanted, expensive surgical procedures). \cite{iversen-luras} and \cite{iversen2004} provide empirical evidence that somewhat supports it: physicians with a shortage of customers will provide more services, thus obtaining more income \textit{per customer}.

To our knowledge, the only article dealing specifically with sick leave certicate granting as a function of competition among physicians is \cite{markussen-roed}. \cite{cln} deal with sick leave as well, but make only the narrower point that, in a Bayesian context, doctors have almost no incentive to distrust patients’ self-reported, unverifiable symptoms. \citeauthor{markussen-roed}’s methodology is similar to our own: after performing ``raw'’ rgeression analysis, they set-up a model of patient choice as a McFadden logit over observables $X_i$, including physician leniency (assumed observable), and as such can estimate the role leniency plays in demand for their services. In a parallel exercise, they conclude, with admittedly unclear causal interpretation (p. 1): ``GPs are more lenientgatekeepers the more competitive is the physician market, and a reputation for lenient gatekeeping increases the demand for their services''.

Despite the different subject matter, the main source of inspiration for this paper is \cite{schnell2017physician}, and can be seen as an attempt to replicate her model and framework, intially devised for opioid markets, to the market for sick leaves. In repurposing her framework it underwent several key transformations, whose impact on physician behavior, in comparison with Schnell’s original paper, will be laid out more clearly in the APPPENDIX.

MORE MENTION OF ESTABLISHED FACTS BY ROED-MARKUSSEN




\end{document}