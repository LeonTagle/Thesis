\documentclass{article}
\usepackage{graphicx} % Required for inserting images
\usepackage{amsmath, amsthm, amssymb}
\usepackage{bbm} %para 1 indicatriz
\usepackage{cancel}
\usepackage{enumitem} %fancy itemize
\usepackage{thmtools} %fancy theorems
\usepackage{bm} %math bold-face

\newtheorem{theorem}{Theorem}[section]
\newtheorem{lemma}[theorem]{Lemma}
\newtheorem{corollary}[theorem]{Corollary}
\newtheorem{definition}[theorem]{Definition}
\newtheorem{ass}[theorem]{Assumption}
\newtheorem{prop}[theorem]{Proposition}

\newcommand\maybegeq{\stackrel{?}{\geq}}

\declaretheoremstyle[
notefont=\bfseries, notebraces={}{},
bodyfont=\normalfont\itshape,
headformat=\NAME \NOTE
]{nopar}
\declaretheorem[style=nopar]{equilibrium}

\renewcommand{\thesubsubsection}{\alph{subsubsection}}
% Subsubsections with .a .b etc

\title{formal}
\author{leontaglelb }
\date{July 2024}

\begin{document}


\section{Note on language}

For ease of reference, we refer to any patient --otherwise called client or worker-- as ``he'', and to any doctor --or physician-- as ``she''.

At this point it shall be noted that the first person plural (we) employed over the course of this article is to be read as a royal \textit{we}, being that we claim sole authorship over this paper.


\section{Model}

We consider $i = 1, ..., I$ patients and $j = 1, ..., J$ doctors.

Patients are characterized by the tuple $(\kappa_i,\gamma_i) \in (\mathbb{R}_0^+)^2$, their ``medical need'' and ``taste for sick leave'' respectively, following the ex-ante cumulative distributions $F(k)$ and $G(\gamma)$.

Doctors are described by their ``service quality'' $V_j \in \mathbb{R}_0^+$, following the \textit{ex-post}, empirical distribution $H(V)$.

A patient $i$ visits a doctor for treatment and may be granted a sick leave certificate. After being assigned a doctor $j$, his utility function --implicitly dependent on his characteristic $(\kappa_i,\gamma_i)$ tuple-- is defined piece-wise as follows:
    \[
U_i(V_j) =\begin{cases}
\gamma_i + V_j \kappa_i - \tau \text{  \hspace{0.65cm} if he’s granted a certificate,} \\
V_j \kappa_i - \tau \text{  \hspace{0.5cm} if he only visits the doctor,} \\
0 \text{  \hspace{0.8cm} if he doesn't see the doctor,}
\end{cases}
\]

As we see, there's three components to patient utility: an interaction between the patient's medical need $\kappa_i$ and the physician's service quality $V_j$ which implies their complimentarity, his ``taste'' for sick leave $\gamma_i$ in the case he's granted one, and $\tau$, the cost of visit, normalized across doctors.

Whereas patients may visit at most one doctor, a doctor may see several physicians. We define $Q_j$ as the expected number of patients of doctor $j$, the demand for her services. We say ``expected'' because, as we'll see later on, patients may opt for a mixed strategy, assigning a certain \textit{probability} to visiting $j$, and $Q_j$ will be defined over the ex-ante probabilities of all patients and not their ex-post realization.

As doctor $j$ has the option to grant a sick leave certificate to a given patient $i$ which visits her, so we likewise define $X_j$ as the \textit{expected} number of such certificates doctor $j$ will dole out, given her ex-ante client demand and how many of them would be granted one.

We now define the physician's utility function as follows:
\[
U_j(Q_j, X_j) = R_j(Q_j) - P_j(X_j)
\]
where $R_j(\cdot)$ is an individual, concave \textit{revenue} function defined over expected total clients, and $P_j(\cdot)$ is a convex \textit{punishment} function on $X_j$, grouping her personal preference as well as institutional incentives. The implication is that after a given number of patients the disutility of an additional certificate issued would outweigh doctor $j$ financial incentives for further clientele.

The $j$ subindex indicates that we will allow $R_j(\cdot)$ and $P_j(\cdot)$ to vary across physicians, meaning that a given doctor $j$ will now be initially characterized by both her parameter $V_j$ as well as the form of her $R_j(\cdot)$ and $P_j(\cdot)$ functions.

We stress again that we define physician utility in terms of the \textit{expected} realization of patient demand and granted sick leaves, as befits the logic of this game, where doctors take action before patients (we will specify the timing of our game below).

Following \cite{schnell2017physician}, we focus on \textit{threshold equilibria}, wherein each physician's strategy is the choice of a value $\bar{\kappa_j}$, such that of the patients who visit $j$, those with a $\kappa_i$ value above or at that threshold will recieve a certificate, and those under it won't.

In both frameworks we will develop, with and without search, the set $\{\bar{\kappa_j}\}_{j =1}^{J}$ composed of all physicians' choice of $\bar{\kappa_j}$ will be public knowledge to patients at the moment they choose their strategy, whereas doctors themselves don't observe it at the moment they make their choice of threshold, because they will select it simultaneously -- and \textit{then} patients make their move.

Our models are different iterations of a general game with the following timing:
\begin{itemize}
    \item First stage: Physicians simultaneously choose $\bar{\kappa_j}$.
    \item Second stage: Observing $\{\bar{\kappa_j}\}_{j =1}^{J}$, each patient chooses or is assigned some doctor $j$.
    \item Third stage: Each patient can choose to see their doctor and incur a visit cost $\tau$, or refrain from doing so, leaving him and his physician with null utility.
\end{itemize}

Conditional upon his visit, the utility of patient $i$ from seeing doctor $j$ is:
\[
    y_i(V_j,\bar{\kappa_j}) =\begin{cases}
    \gamma_i + V_j \kappa_i - \tau \text{  \hspace{0.65cm} if $\kappa_i \geq \bar{\kappa_j}$,} \\
    V_j \kappa_i - \tau \text{  \hspace{0.5cm} if $\kappa_i < \bar{\kappa_j}$,}
    \end{cases}
\]

Patient utility $U_i$ as previously defined, now taking $\bar{\kappa_j}$ explicitly as an argument as well as $V_j$, can be expressed as a left-censored function over $y_i$:
\[
U_i(V_j,\bar{\kappa_j}) =  \operatorname{max} \{y_i(V_j,\bar{\kappa_j}),0\}
\]

\subsection{Non-search Equilibrium}

We first devote attention to a non-search baseline, where all patients are randomly, symmetrically assigned to a physician, and their only say in the matter is whether they’ll then visit doctor $j$, i.e. the second stage of the game is out of their hands, and they only make choices in the third stage after assingment.

A patient won’t visit his assigned physician if his expected utility from such a visit is negative, we call this a \textit{free disposal} requirement. As such, a doctor $j$’s expected client demand, as a function of $\bar{\kappa_j}$ and given the parameter $V_j$, will be the following:
\begin{equation}
    Q_j(\bar{\kappa_j}) \,=\, \frac{I}{J}\left[ \int_{\tau/V_j}^{\infty}\,dF(k) +  \int_{\min\{\bar{\kappa_j},\tau/V_j\}}^{\tau/V_j} \int_{\tilde{\gamma}(k)}^{\infty} \,dG(\gamma) \,dF(k) \right] \tag{N.1}\label{eq:ns_Q}
\end{equation}
where the left term consists of the mass of patients who just by virtue of doctor $j$’s service quality $V_j$ would be willing to pay a visit (i.e. $\kappa_i \geq \tau/V_j$), and the right term would be patients who only see doctor $j$ out of the expectation of getting sick leave ($\kappa_i \geq \bar{\kappa_j}$ \& $\gamma_i \geq \tau - V_j \kappa_i$), but wouldn’t visit otherwise. We define $\tilde{\gamma}(k) := \tau - V_j k$ as the lower limit of the inner integral.

Given that each patient with a $\kappa_i$ higher or equal to $\bar{\kappa_j}$ is granted a sick leave certificate, the expected total number of such certificates granted by $j$, as a function of $\bar{\kappa_j}$, is:

\begin{equation}
    X_j(\bar{\kappa_j}) \,=\, \frac{I}{J} \int_{\bar{\kappa_j}}^{\infty} \int_{\tilde{\gamma}(k)}^{\infty} \,dG(\gamma) \,dF(k)
    \tag{N.2}\label{eq:ns_X}
\end{equation}

Given (\ref{eq:ns_Q}) and (\ref{eq:ns_X}), each physician solves for the following unconstrained optimization:
\begin{equation}
\bar{\kappa_j}^* \equiv \max_{\bar{\kappa_j}} R_j(Q_j) - P_j(X_j)
\tag{N.3}\label{eq:ns_k}
\end{equation}

\textit{MÁS SUPUESTOS?}

\begin{lemma}
\label{ns_lemma}
No value in $(\frac{\tau}{V_j},\infty)$ can be the optimal solution of (\ref{eq:ns_k}).
\end{lemma}

PROOF: APPENDIX

\begin{prop}
    \label{ns_derivatives}
When $\bar{\kappa_j} \in [0,\frac{\tau}{V_j}]$,
\[
    \frac{\partial Q_j}{\partial\bar{\kappa_j}} = \frac{\partial X_j}{\partial \bar{\kappa_j}}   
\]
\end{prop}

This proposition implies that when $\bar{\kappa_j}$ is low enough that the marginal patient is indifferent, any new clients gained by doctor $j$ are those she entices through the expectation of getting a certificate, meaning that in the vicinity of $\bar{\kappa_j}$ the change in expected patients $\Delta Q_j/\Delta\bar{\kappa_j}$ through a variation in $\bar{\kappa_j}$ is the same as the change in expected certificates granted $\Delta X_j/\Delta \bar{\kappa_j}$.

Given Proposition \ref{ns_derivatives}, the following equation, which is the FOC of (\ref{eq:ns_k}):
\[
R_j^{\prime}(Q_j)\frac{\partial Q_j}{\partial\bar{\kappa_j}}  - P_j^{\prime}(X_j)\frac{\partial X_j}{\partial \bar{\kappa_j}} = 0
\]
may be simplified into
\begin{equation}
    R^{\prime}(Q_j) = P^{\prime}(X_j)
    \tag{N.4}\label{eq:ns_FOC}
\end{equation}

We can see that (\ref{eq:ns_k}) offers three possibilities by way of solution: a \textit{corner} solution towards $\infty$, where doctor $j$ is content with her ``captured'' clientele (those who would visit her even without expecting sick leave), and offers no certificates at all\footnote{We may rationalize this by interpreting it as physician $j$ being as strict as is medically responsible. We could have written this as $\bar{\kappa_j}$ having an upper bound $\kappa_{\operatorname{max}}$, where any physician would be obliged to grant sick leave to a patient with $\kappa_i > \kappa_{\operatorname{max}}$.}; the corner solution $\bar{\kappa_j} = 0$, maximum leniency; and an \textit{inner} solution in $[0,\frac{\tau}{V_j}]$ fulfilling equation (\ref{eq:ns_FOC}) and thus the FOC of (\ref{eq:ns_k}). We formalize equilibrium below.

\begin{equilibrium}[in non-search models]
    \label{ns_eq}
Given a doctor-patient market specified by $(\{(\kappa_i,\gamma_i)\}_{i =1}^{I},\{(V_j,R_j(\cdot),P_j(\cdot))\}_{i =1}^{J})$, we define an equilibrium as a set of physician thresholds $\{\bar{\kappa_j}\}_{j =1}^{J}$ satisfying (\ref{eq:ns_k}) and their corresponding equilibrium (expected) patient demand and granted medical licenses $\{(Q_j,X_j)\}_{j =1}^{J}$ as described by (\ref{eq:ns_Q}), (\ref{eq:ns_X}), respectively.
\end{equilibrium}

\subsection{Search Equilibria}

\subsubsection{Introducing search}

We introduce patient ``search'' as a general framework in which patients can choose freely among all physicians.

We define for each patient $i$ a vector $S_i \in \Delta(\mathcal{J})$, where $\mathcal{J}$ is the the $1 \times J$ vector composed of all $1, ..., J$ physicians. $S_i$ will be his \textit{strategy} for this game, representing his probabilistic choice of visit for each doctor $j$, such that each component $s_{i1}, ... , s_{iJ}$ of $S_i$ stands for the probability that he'll visit doctors $1, ..., J$ respectively.

In order to describe a proper probability distribution, the following criteria must be met:


\begin{enumerate}[label=\roman*.]
    \item $\forall j, \, s_{ij} \geq 0$
    \item $\sum_{i = 1}^{J} s_{ij} \leq 1$
\end{enumerate}

We will allow the sum of all components to be less than one, implying the presence of an \textit{outside option} for patients, that is, to not visit any doctors. Such an option is important, as patient rationality in our models will include  ``\textit{free disposal}'', meaning that a patient will never visit a doctor if his expected utility from such a visit is less than $0$, i.e. $s_{ij} = 0$ if $U_i(V_j,\bar{\kappa_j}) = 0$. This makes the third stage trivial, as under free choice, patient $i$ will only be assigned with positive probability to doctors he will be willing to visit.

We can re-interpret the non-search model as each patient being made to play by the strategy \{$S_i: s_{i1} = s_{i2} ... = s_{iJ} = \frac{1}{J}$ \}, and it's the lack of a free choice which makes the third stage non-trivial.

Given the collection of all patients' strategies $(\{S_i\}_{i =1}^{I}$, each with a respective $s_{ij}$ for doctor $j$, we can formulate expected clientele and granted sick leaves for $j$ in a manner very much alike the previous section (indeed, as a generalization):
\begin{equation}
    Q_j(\bar{\kappa_j}, \bar{\kappa}_{-j}) \,=\, \int_{\tau/V_j}^{\infty}\bm{s_{ij}}\,dF(k) +  \int_{\min\{\bar{\kappa_j},\tau/V_j\}}^{\tau/V_j} \int_{\tilde{\gamma}(k)}^{\infty}\bm{s_{ij}} \,dG(\gamma) \,dF(k)  \tag{S.1}\label{eq:s_Q}
\end{equation}

\begin{equation}
    X_j(\bar{\kappa_j}, \bar{\kappa}_{-j}) \,=\, \int_{\bar{\kappa_j}}^{\infty} \int_{\tilde{\gamma}(k)}^{\infty}\bm{s_{ij}} \,dG(\gamma) \,dF(k)
    \tag{S.2}\label{eq:s_X}
\end{equation}

As before, the left term in equation (\ref{eq:s_Q}) includes patients who would visit $j$ even without sick leave, whereas the right term those who only do so because they expect one.

Notice that we now present $Q_j$ and $X_j$  not only in terms of the threshold $\bar{\kappa_j}$ chosen by doctor $j$, but also in terms of the thresholds of the other $J - 1$ doctors, which we abreviate as $\bar{\kappa}_{-j}$. Whereas before doctors were simply alloted a given number of patients, now they will \textit{compete} for them, as patients' $S_i$ strategy will consider the whole of $(\{\bar{\kappa_j}\}_{j =1}^{J}$ when considering which physician(s) they will visit with positive probability.

\subsubsection{Some analytic results}

We shall further specify the form $S_i$ will take. Consider now a $1 \times J$ vector $U_i$, where each component $u_{i1}, ... , u_{iJ}$ indicates the utility patient $i$ expects from a visit to doctors $1, ..., J$ respectively, corresponding to $U_i(V_j, \bar{\kappa_j})$ for each doctor $j$. As a shorthand, we shall write $u_{i,-j}$ to indicate the $J - 1$ components of $U_i$ excluding $u_{ij}$.

In the models considered, each component $s_{ij}$ of $S_i$ will be defined as:
\[s_{ij} \equiv g_i(u_{ij}, u_{i,-j})\]
where $g_i(\cdot)$ is a continuous function weakly increasing in the first argument $u_{ij}$, and weakly decreasing in the remaining arguments given by $u_{i,-j}$. Our model specifications will consist in giving this function $g_i(u_{ij}, u_{i,-j})$ a specific form.

As we show in the APPENDIX, such prerrequisites over $s_{ij}$ are enough to prove most of the qualitative relations we expect to see regarding equilibrium aggregates and doctor strategies, but not enough to prove that $\bar{\kappa_j}$ and $\bar{\kappa_{-j}}$ are strategic complements -- i.e., that for any $l \neq j$, $\frac{\partial^2 U_{j}}{\partial\bar{\kappa_{j}} \partial\bar{\kappa_{l}}} \geq 0$

\subsubsection{The ``implicit'' search model}



\end{document}