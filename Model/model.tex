\documentclass{article}
\usepackage{graphicx} % Required for inserting images
\usepackage{amsmath}
\usepackage{amssymb} % Fancy R
\usepackage{bbm} %para 1 indicatriz
\usepackage{cancel}
\usepackage{enumitem} %fancy itemize
\usepackage{apacite} %bibliography style

\title{formal}
\author{leontaglelb }
\date{July 2024}

\begin{document}


\section{Model}

We consider $i = 1, ..., I$ patients and $j = 1, ..., J$ doctors.

Patients are characterized by the tuple $(\kappa_i,\gamma_i) \in (\mathbb{R}_0^+)^2$, their ``medical need'' and ``taste for licenses'' respectively, following the ex-ante cumulative distributions $F(k)$ and $G(\gamma)$.

Doctors are described by their ``service quality'' $V_j \in \mathbb{R}_0^+$, following the \textit{ex-post}, empirical distribution $H(V)$.

A patient $i$ visits a doctor for treatment and may be granted a medical license. As such, their utility function --implicitly dependent on his $(\kappa_i,\gamma_i)$ tuple-- is defined piece-wise as follows:
    \[
U_i(V_j) =\begin{cases}
\gamma_i + V_j \kappa_i - \tau \text{  \hspace{0.65cm} if he’s granted a license,} \\
V_j \kappa_i - \tau \text{  \hspace{0.5cm} if he only visits a doctor,} \\
0 \text{  \hspace{0.8cm} if he doesn't see a doctor,}
\end{cases}
\]

As we see, there's three components to patient utility: an interaction between the patient's medical need $\kappa_i$ and the physician's service quality $V_j$ which implies their complimentarity, his taste for licenses $\gamma_i$ in the case he’s granted one, and $\tau$, the cost of visit, normalized across doctors.


\end{document}