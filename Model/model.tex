\documentclass{article}
\usepackage{graphicx} % Required for inserting images
\usepackage{amsmath}
\usepackage{amssymb} % Fancy R
\usepackage{bbm} %para 1 indicatriz
\usepackage{cancel}
\usepackage{enumitem} %fancy itemize
\usepackage{apacite} %bibliography style

\title{formal}
\author{leontaglelb }
\date{July 2024}

\begin{document}


\section{Note on language}

For ease of reference, we refer to any patient --otherwise called client or worker-- as ``he'', and to any doctor --or physician-- as ``she''.

At this point it shall be noted that the first person plural (we) employed over the course of this article is to be read as a royal \textit{we}, being that we claim sole authorship over this paper.


\section{Model}

We consider $i = 1, ..., I$ patients and $j = 1, ..., J$ doctors.

Patients are characterized by the tuple $(\kappa_i,\gamma_i) \in (\mathbb{R}_0^+)^2$, their ``medical need'' and ``taste for sick leave'' respectively, following the ex-ante cumulative distributions $F(k)$ and $G(\gamma)$.

Doctors are described by their ``service quality'' $V_j \in \mathbb{R}_0^+$, following the \textit{ex-post}, empirical distribution $H(V)$.

A patient $i$ visits a doctor for treatment and may be granted a sick leave certificate. As such, their utility function --implicitly dependent on his characteristic $(\kappa_i,\gamma_i)$ tuple-- is defined piece-wise as follows:
    \[
U_i(V_j) =\begin{cases}
\gamma_i + V_j \kappa_i - \tau \text{  \hspace{0.65cm} if he’s granted a certificate,} \\
V_j \kappa_i - \tau \text{  \hspace{0.5cm} if he only visits a doctor,} \\
0 \text{  \hspace{0.8cm} if he doesn't see a doctor,}
\end{cases}
\]

As we see, there's three components to patient utility: an interaction between the patient's medical need $\kappa_i$ and the physician's service quality $V_j$ which implies their complimentarity, his ``taste'' for sick leave $\gamma_i$ in the case he’s granted one, and $\tau$, the cost of visit, normalized across doctors.

Whereas patients may visit at most one doctor, a doctor may see several physicians. We define $Q_j$ as the expected number of patients of doctor $j$, the demand for her services. We say ``expected'' because, as we'll see later on, patients may opt for a mixed strategy, assigning a certain \textit{probability} to visiting $j$, and $Q_j$ will be defined over the ex-ante probabilities of all patients and not their ex-post realization.

As doctor $j$ has the option to grant a sick leave certificate to a given patient $i$ which visits her, so we likewise define $X_j$ as the \textit{expected} number of such certificates doctor $j$ will dole out, given her ex-ante client demand and how many of them would be granted one.

We now define the physician's utility function as follows:
\[
U_j(Q_j, X_j) = R_j(Q_j) - P_j(X_j)
\]
where $R_j(\cdot)$ is an individual, concave \textit{revenue} function defined over expected total clients, and $P_j(\cdot)$ is a convex \textit{punishment} function on $X_j$, grouping her personal preference as well as institutional incentives. The implication is that after a given number of patients the disutility of an additional certificate issued would outweigh doctor $j$ financial incentives for further clientele.

Following \cite{schnell2017physician}, we focus on \textit{threshold equilibriums}, wherein each physician’s strategy is the choice of a value $\bar{\kappa_j}$, such that of the patients who visit $j$, those with a $\kappa_i$ value above or at that threshold will recieve a certificate, and those under it won’t.

In both frameworks we will work with, with and without search, the vector $\bar{\kappa}$ composed of all physicians’ choice of $\bar{\kappa_j}$ will be public knowledge to patients at the moment they choose their strategy, whereas doctors themselves don’t observe it at the moment they make their choice of threshold, because they will select it simultaneously -- and \textit{then} patients make their move.

\subsection{Non-search Equilibrium}

We first devote attention to a non-search baseline, where all patients are randomly, symmetrically assigned to a physician, and their only say in the matter is whether they’ll then visit doctor $j$.

A patient won’t visit his assigned physician if his expected utility from such a visit is negative, we call this a \textit{free disposal} requirement. As such, a doctor $j$’s expected client demand, as a function of $\bar{\kappa_j}$ and given the parameter $V_j$, will be the following:
\begin{equation}
    Q_j(\bar{\kappa_j}) \,=\, \frac{I}{J}\left[ \int_{\tau/V_j}^{\infty}\,dF(k) +  \int_{\min\{\bar{\kappa_j},\tau/V_j\}}^{\tau/V_j} \int_{\tilde{\gamma}(k)}^{\infty} \,dG(\gamma) \,dF(k) \right] \tag{N.1}\label{eq:N.1}
    \end{equation}
where the left term envelops those patients who just by virtue of doctor $j$’s service quality $V_j$ would be willing to pay a visit (i.e. $\kappa_i \geq \tau/V_j$), and the right term would be patients who only see doctor $j$ out of the expectation of getting sick leave ($\kappa_i \geq \bar{\kappa_j}$ \& $\gamma_i \geq \tau - V_j \kappa_i$), but wouldn’t visit otherwise. We define $\tilde{\gamma}(k) := \tau - V_j k$ as the lower limit of the inner integral.

Given that each patient with a $\kappa_i$ higher or equal to $\bar{\kappa_j}$ is granted a sick leave certificate, the expected total number of such certificates granted by $j$, as a function of $\bar{\kappa_j}$, is:

\begin{equation}
    X_j(\bar{\kappa_j}) \,=\, \frac{I}{J} \int_{\bar{\kappa_j}}^{\infty} \int_{\tilde{\gamma}(k)}^{\infty} \,dG(\gamma) \,dF(k)
    \tag{N.2}\label{eq:N.2}
\end{equation}

Given (\ref{eq:N.1}) and (\ref{eq:N.2}), each physician solves for the following unconstrained optimization:
\[
\bar{\kappa_j}^* \equiv \max_{\bar{\kappa_j}} R_j(Q_j) - P_j(X_j)
\]

LEMMA: No value in $(\frac{\tau}{V_j},\infty)$ can be the optimal solution of $\bar{\kappa_j}^*$.

PROOF: APPENDIX

\end{document}